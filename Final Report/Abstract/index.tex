\section*{Final Report Plagiarism Statement}
I affirm that I have submitted, or will submit, an electronic copy of 
my final year project report to the provided EEE link.\\ \newline \noindent I affirm 
that I have provided explicit references for all the material in my Final Report that 
is not authored by me, but is represented as my own work.

\newpage


\section*{Abstract (CHANGE AT END)}
%The abstract should (very concisely) summarise the
%topic, content and conclusions of the project.
%Abstracts can vary in length from 50 words up to at
%most 200 words. They are more concise than
%executive summaries. \\ \newline \noindent The abstract is a very brief summary of
%the report's contents. It should be about half a
%page long. Somebody unfamiliar with your
%project should have a good idea of what it's
%about having read the abstract alone and will
%know whether it will be of interest to them.

\textcolor{red}{Abstract should consist of motivation, methods, results, conclusion.
All concise \& capturing the reader within 200 words. Your abstract needs working and finalising once you have your results and concluding statement. }


The prevention,  evaluation,  and  treatment  of  hypertension  
have  attracted  increasing  attention  in  recent  years. The advancement of wearable technology has resulted in increasing importance 
into the monitoring of non-invasive ambulatory blood pressure, compared to the traditional 
invasive blood-pressure monitoring methods. As photoplethysmography (PPG) technology has been widely 
applied to wearable sensors, the noninvasive estimation of blood
pressure (BP) using the PPG method has received considerable interest. For this project, systolic and diastolic BPs are estimated using PPG signals. A Recurrent Neural Network (RNN) is 
used for estimation. Due to their being several alternative existing methods for 
estimating blood pressure, it was necessary to perform a comparison between 
the best performing Deep Learning based methods. Overall, the proposed 
method obtains better accuracy. The model achieves a mean absolute error of  mmHg 
for systolic BP and  mmHg for diastolic BP.