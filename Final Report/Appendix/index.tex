\section*{Appendix}
\begin{comment}
   The appendices contain information
that is peripheral to the main body of the report.
Information typically included are things like
program listings, complex circuit diagrams,
tables, proofs, graphs or any other material
which would break up the theme of the text if it
appeared in situ. Large program listings may be
submitted with the report although it is
preferable either to provide them on CD, or to
cite details of a suitable accessible cloud
repository containing the material. Where CDs
are used you must prepare two CDs, one for each
paper copy of the report.\\ \newline \noindent Please use 
appendices as necessary for material that
is tangentially relevant, or necessary to preserve
project value, but that you do not expect Examiners to
read. Note that software projects with significant code
should normally provide electronic versions of the
code on USB stick, CDROM, or cloud repository. In
that case the Appendix should state where the code
may be found. \textbf{The Appendix should normally
include, or refer to, all the technical details needed
for another user to continue code development}.
Typical items that should not be in the main report,
but should (possibly - since Appendices are not
normally read this is a matter of judgement) be in
appendices:
\begin{itemize}
    \item Source code. Note that code (if long) should
    also be available in some electronic form, e.g. a github repository, CD, or USB 
    stick. Software
    projects must provide access to the source
    code in some form. The Appendix will then
    reference this.
    \item Test data sets (again, for large volumes of
    data an electronic form would be more
    appropriate). The report itself will contain
    concise summaries of the test data in a
    human readable form.
    \item Raw results. Tables of results in unreadable
    form should not be put in the main report, but
    if needed may be put in an Appendix.
    \item Related material possibly but not directly
    relevant to the project work. E.g. manuals of
    test equipment used. (There is no
    requirement to include such, but in some
    cases, where they have some tangential
    relevance, it might be appropriate)
\end{itemize}
 
\end{comment}
\subsection*{FYP Mission Statement (correct to June 2022)}
'\emph{Ambulatory blood pressure monitoring has become increasingly 
relevant due to the advancement of wearable technology. 
Modalities such as Electrocardiogram (ECG) and Photoplethysmography 
(PPG) have provided an indirect method of blood pressure estimations 
compared to a traditional blood-pressure monitor. Although algorithms 
exist for calculating blood pressure with ECG and PPG, it is vital 
that their computational complexity is minimal, whilst maintaining 
accuracy, due to the power limitations of wearable technology. The 
goal of this project is to establish the tradeoffs between using 
PPG and ECG to quantify blood pressure, in the context of wearable
 technology, where power must be kept to a minimum. This project is
  ideal for students interested in signal processing, who have 
  excellent programming skills in Matlab.}'

\subsection*{Python code for the neural network architectures}
\subsubsection*{AlexNet}
\begin{python}    
    from tensorflow.keras.layers import Softmax, Permute, Input, Add, Conv1D, MaxPooling1D, Dense, Activation, ZeroPadding2D, BatchNormalization, Flatten, Conv2D, AveragePooling1D, MaxPooling2D, GlobalMaxPooling2D, LeakyReLU, GlobalAveragePooling2D, ReLU, Dropout
    from tensorflow.keras.initializers import glorot_uniform
    from tensorflow.keras.models import Model
    import tensorflow as tf
    import scipy
    
    def AlexNet_1D(data_in_shape, num_output=2, dil=1, kernel_size=3, fs = 125, useMaxPooling=True, UseDerivative=False):
    
        # Define the input as a tensor with shape input_shape
        X_input = Input(shape=data_in_shape)
    
        if UseDerivative:
            dt1 = (X_input[:,1:] - X_input[:,:-1])*fs
            dt2 = (dt1[:,1:] - dt1[:,:-1])*fs
            dt1 = tf.pad(dt1, tf.constant([[0,0],[0,1],[0,0]]))
            dt2 = tf.pad(dt2, tf.constant([[0,0],[0,2],[0,0]]))
            X = tf.concat([X_input, dt1, dt2], axis=2)
        else:
            X=X_input
    
    
        # convolutional stage
        X = Conv1D(96, 7, strides=3, name='conv1', kernel_initializer=glorot_uniform(seed=0), padding="same")(X)
        if useMaxPooling:
            X = MaxPooling1D(3, strides=2, name="MaxPool1")(X)
        X = Activation(ReLU())(X)
        X = BatchNormalization(axis=-1, name='BatchNorm1')(X)
    
        X = Conv1D(256, kernel_size=kernel_size, strides=1, dilation_rate=dil, name='conv2', kernel_initializer=glorot_uniform(seed=0), padding="same")(X)
        if useMaxPooling:
            X = MaxPooling1D(3, strides=2, name="MaxPool2")(X)
        X = Activation(ReLU())(X)
        X = BatchNormalization(axis=-1, name='BatchNorm2')(X)
    
        X = Conv1D(384, kernel_size=kernel_size, strides=1, dilation_rate=dil, name='conv3', kernel_initializer=glorot_uniform(seed=0), padding="same")(X)
        X = Activation(ReLU())(X)
        X = BatchNormalization(axis=-1, name='BatchNorm3')(X)
    
        X = Conv1D(384, kernel_size=kernel_size, strides=1, dilation_rate=dil, name='conv4', kernel_initializer=glorot_uniform(seed=0), padding="same")(X)
        X = Activation(ReLU())(X)
        X = BatchNormalization(axis=-1, name='BatchNorm4')(X)
    
        X = Conv1D(256, kernel_size=kernel_size, strides=1, dilation_rate=dil, name='conv5', kernel_initializer=glorot_uniform(seed=0), padding="same")(X)
        if useMaxPooling:
            X = MaxPooling1D(3, strides=2, name="MaxPool5")(X)
        X = Activation(ReLU())(X)
        X = BatchNormalization(axis=-1, name='BatchNorm5')(X)
    
        # Fully connected stage
        X = Flatten()(X)
        X = Dense(4096, activation='relu', name='dense1', kernel_initializer=glorot_uniform(seed=0))(X)
        X = Dropout(rate=0.5)(X)
        X = Dense(4096, activation='relu', name='dense2', kernel_initializer=glorot_uniform(seed=0))(X)
        X = Dropout(rate=0.5)(X)
    
        # Create model
        if num_output == 1:
            X_out = Dense(3, activation='softmax', name='out', kernel_initializer=glorot_uniform(seed=0))(X)
            model = Model(inputs=X_input, outputs=X_out, name='AlexNet_1D')
        else:
            # output stage
            X_SBP = Dense(1, activation='relu', name='SBP', kernel_initializer=glorot_uniform(seed=0))(X)
            X_DBP = Dense(1, activation='relu', name='DBP', kernel_initializer=glorot_uniform(seed=0))(X)
            model = Model(inputs=X_input, outputs=[X_SBP, X_DBP], name='AlexNet_1D')
    
        return model
\end{python}

\subsubsection*{ResNet}
\begin{python}
from tensorflow.keras import layers
from tensorflow.keras.layers import Input, Add, Dense, Activation, ZeroPadding1D, BatchNormalization, Flatten, Conv1D, AveragePooling1D, MaxPooling1D, GlobalMaxPooling2D, LeakyReLU, GlobalAveragePooling2D, ReLU, concatenate
from tensorflow.keras.initializers import glorot_uniform, constant
from tensorflow.keras.models import Model
import tensorflow as tf


def identity_block(X, f, filters, stage, block, dil=1):

    # Implementation of the identity block as defined in Figure 3
    
    # Arguments:
    # X -- input tensor of shape (m, n_H_prev, n_W_prev, n_C_prev)
    # f -- integer, specifying the shape of the middle CONV's window for the main path
    # filters -- python list of integers, defining the number of filters in the CONV layers of the main path
    # stage -- integer, used to name the layers, depending on their position in the network
    # block -- string/character, used to name the layers, depending on their position in the network
    
    #  Returns:
    # X -- output of the identity block, tensor of shape (n_H, n_W, n_C)
    
    # defining name basis
    conv_name_base = 'res' + str(stage) + block + '_branch'
    bn_name_base = 'bn' + str(stage) + block + '_branch'
    
    # Retrieve Filters
    F1, F2, F3 = filters
    
    # Save the input value. You'll need this later to add back to the main path. 
    X_shortcut = X
    
    # First component of main path
    X = Conv1D(filters = F1, kernel_size = 1, strides = 1, dilation_rate=dil, padding = 'valid', name = conv_name_base + '2a', kernel_initializer = glorot_uniform(seed=0))(X)
    X = BatchNormalization(momentum = 0.9, name = bn_name_base + '2a')(X)
    X = Activation(ReLU())(X)

    # Second component of main path
    X = Conv1D(filters = F2, kernel_size = f, strides = 1, dilation_rate=dil, padding = 'same', name = conv_name_base + '2b', kernel_initializer = glorot_uniform(seed=0))(X)
    X = BatchNormalization(momentum=0.9, name=bn_name_base + '2b')(X)
    X = Activation(ReLU())(X)

    # Third component of main path
    X = Conv1D(filters = F3, kernel_size = 1, strides = 1, dilation_rate=dil, padding = 'valid', name = conv_name_base + '2c', kernel_initializer = glorot_uniform(seed=0))(X)
    X = BatchNormalization(momentum = 0.9, name = bn_name_base + '2c')(X)

    # Final step: Add shortcut value to main path, and pass it through a RELU activation
    X = Add()([X, X_shortcut])
    X = Activation(ReLU())(X)
    # X = BatchNormalization(momentum = 0.9, name = bn_name_base + '2c')(X)
    
    return X

def convolutional_block(X, f, filters, stage, block, s = 2, dil = 1):

    # Implementation of the convolutional block as defined in Figure 4
    
    # Arguments:
    # X -- input tensor of shape (m, n_H_prev, n_W_prev, n_C_prev)
    # f -- integer, specifying the shape of the middle CONV's window for the main path
    # filters -- python list of integers, defining the number of filters in the CONV layers of the main path
    # stage -- integer, used to name the layers, depending on their position in the network
    # block -- string/character, used to name the layers, depending on their position in the network
    # s -- Integer, specifying the stride to be used
    
    # Returns:
    # X -- output of the convolutional block, tensor of shape (n_H, n_W, n_C)
    
    # defining name basis
    conv_name_base = 'res' + str(stage) + block + '_branch'
    bn_name_base = 'bn' + str(stage) + block + '_branch'
    
    # Retrieve Filters
    F1, F2, F3 = filters
    
    # Save the input value
    X_shortcut = X


    ##### MAIN PATH #####
    # First component of main path 
    X = Conv1D(F1, 1, strides = s, name = conv_name_base + '2a', kernel_initializer = glorot_uniform(seed=0))(X)
    X = BatchNormalization(momentum=0.9, name=bn_name_base + '2a')(X)
    X = Activation(ReLU())(X)
    
    # Second component of main path
    X = Conv1D(filters = F2, kernel_size = f, strides = 1, dilation_rate=dil, padding = 'same', name = conv_name_base + '2b', kernel_initializer = glorot_uniform(seed=0))(X)
    X = BatchNormalization(momentum=0.9, name=bn_name_base + '2b')(X)
    X = Activation(ReLU())(X)

    # Third component of main path
    X = Conv1D(filters = F3, kernel_size = 1, dilation_rate=dil, strides = 1, padding = 'valid', name = conv_name_base + '2c', kernel_initializer = glorot_uniform(seed=0))(X)
    X = BatchNormalization(momentum = 0.9, name = bn_name_base + '2c')(X)

    ##### SHORTCUT PATH ####
    X_shortcut = Conv1D(filters = F3, kernel_size = 1, strides = s, padding = 'valid', name = conv_name_base + '1',
                        kernel_initializer = glorot_uniform(seed=0))(X_shortcut)
    X_shortcut = BatchNormalization( momentum = 0.9, name = bn_name_base + '1')(X_shortcut)

    # Final step: Add shortcut value to main path, and pass it through a RELU activation
    X = Add()([X, X_shortcut])
    X = Activation(ReLU())(X)
    # X = BatchNormalization(momentum = 0.9, name = bn_name_base + '2c')(X)
    
    return X

def ResNet50_1D(data_in_shape, num_output=2, fs=125, UseDerivative=False):

    # Implementation of the popular ResNet50 the following architecture:
    # CONV2D -> BATCHNORM -> RELU -> MAXPOOL -> CONVBLOCK -> IDBLOCK*2 -> CONVBLOCK -> IDBLOCK*3
    # -> CONVBLOCK -> IDBLOCK*5 -> CONVBLOCK -> IDBLOCK*2 -> AVGPOOL -> TOPLAYER

    # Arguments:
    # input_shape -- shape of the images of the dataset
    # classes -- integer, number of classes
    # Returns:
    # model -- a Model() instance in Keras

    # Define the input as a tensor with shape input_shape
    X_input = Input(shape=data_in_shape)

    if UseDerivative:
        dt1 = (X_input[:,1:] - X_input[:,:-1])*fs
        dt2 = (dt1[:,1:] - dt1[:,:-1])*fs

        dt1 = tf.pad(dt1, tf.constant([[0,0],[0,1],[0,0]]))
        dt2 = tf.pad(dt2, tf.constant([[0,0],[0,2],[0,0]]))

        X = tf.concat([X_input, dt1, dt2], axis=2)
    else:
        X=X_input

    # Zero-Padding
    X = ZeroPadding1D(3)(X_input)

    # Stage 1
    X = Conv1D(64, 7, strides=2, name='conv1', kernel_initializer=glorot_uniform(seed=0))(X)
    X = BatchNormalization(axis=2, name='bn_conv1')(X)
    X = Activation('relu')(X)
    X = MaxPooling1D(3, strides=3)(X)

    # Stage 2
    X = convolutional_block(X, f=3, filters=[64, 64, 256], stage=2, block='a', s=1)
    X = identity_block(X, 3, [64, 64, 256], stage=2, block='b')
    X = identity_block(X, 3, [64, 64, 256], stage=2, block='c')

    # Stage 3
    X = convolutional_block(X, f = 3, filters = [128, 128, 512], stage = 3, block='a', s = 2)
    X = identity_block(X, 3, [128, 128, 512], stage=3, block='b')
    X = identity_block(X, 3, [128, 128, 512], stage=3, block='c')
    X = identity_block(X, 3, [128, 128, 512], stage=3, block='d')

    # Stage 4
    X = convolutional_block(X, f = 3, filters = [256, 256, 1024], stage = 4, block='a', s = 2)
    X = identity_block(X, 3, [256, 256, 1024], stage=4, block='b')
    X = identity_block(X, 3, [256, 256, 1024], stage=4, block='c')
    X = identity_block(X, 3, [256, 256, 1024], stage=4, block='d')
    X = identity_block(X, 3, [256, 256, 1024], stage=4, block='e')
    X = identity_block(X, 3, [256, 256, 1024], stage=4, block='f')

    # Stage 5
    X = convolutional_block(X, f = 3, filters = [512, 512, 2048], stage = 5, block='a', s = 2)
    X = identity_block(X, 3, [512, 512, 2048], stage=5, block='b')
    X = identity_block(X, 3, [512, 512, 2048], stage=5, block='c')

    X = AveragePooling1D(2, name="avg_pool")(X)

    # output layer
    X = Flatten()(X)
    X_SBP = Dense(1, activation='linear', name='SBP', kernel_initializer = glorot_uniform(seed=0))(X)
    X_DBP = Dense(1, activation='linear', name='DBP', kernel_initializer = glorot_uniform(seed=0))(X)
    HR = Dense(1, activation='linear', name='HR', kernel_initializer = glorot_uniform(seed=0))(X)
    
    # Create model
    if num_output==3:
        model = Model(inputs=X_input, outputs=[X_SBP, X_DBP, HR], name='ResNet50_1D')
    else:
        model = Model(inputs = X_input, outputs = [X_SBP, X_DBP], name='ResNet50_1D')

    return model
\end{python}

\subsubsection*{ResNet with Leave One Subject Out (LOSO)}
\begin{python}
from __future__ import division, print_function
import numpy as np
np.random.seed(3)
import os

from scipy.signal import butter, lfilter, lfilter_zi, filtfilt, savgol_filter
from sklearn.preprocessing import MinMaxScaler
import sklearn as sk
import random
random.seed(3)

import scipy.io as sio
import matplotlib.pyplot as plt
import natsort as natsort
from scipy import signal
import math

import tensorflow as tf
# from tensorflow.keras.utils import multi_gpu_model

from tensorflow.keras.models import Sequential
from tensorflow.keras.backend import squeeze
from kapre import STFT, Magnitude, MagnitudeToDecibel
# from kapre.utils import Normalization2D
from tensorflow.keras.layers import Input, BatchNormalization, Lambda, AveragePooling2D, Flatten, Dense, Conv1D, Activation, add, AveragePooling1D, Dropout, Permute, concatenate, MaxPooling1D, LSTM, Reshape, GRU
from tensorflow.keras.regularizers import l2
from tensorflow.keras import Model
from tensorflow.keras import optimizers
#from tensorflow.keras.utils.vis_utils import plot_model

from tensorflow.keras.layers import Conv2D, MaxPooling2D

def diff(input, fs):
    dt = (input[:, 1:] - input[:, :-1]) * fs
    dt = tf.pad(dt, tf.constant([[0, 0], [0, 1], [0, 0]]))

    return dt

def mid_spectrogram_layer(input_x):
    l2_lambda = .001
    n_dft = 128
    n_hop = 64
    fmin = 0.0
    fmax = 50 / 2

    x = Permute((2, 1))(input_x)
    # x = input_x
    x = STFT(n_fft=n_dft, hop_length=n_hop, output_data_format='channels_last')(x)
    x = Magnitude()(x)
    #x = MagnitudeToDecibel()(x)
    #x = BatchNormalization()(x)
    # x = Normalization2D(str_axis='batch')(x)
    x = Flatten()(x)
    x = Dense(32, activation="relu", kernel_regularizer=l2(l2_lambda))(x)
    x = BatchNormalization()(x)

    return x


def mid_spectrogram_LSTM_layer(input_x):
    l2_lambda = .001
    n_dft = 64

    n_hop = 64
    fmin = 0.0
    fmax = 50 / 2

    #x = Permute((2, 1))(input_x)
    x = input_x
    x = STFT(n_fft=n_dft, hop_length=n_hop, output_data_format='channels_last')(x)
    x = Magnitude()(x)
    x = MagnitudeToDecibel()(x)
   #x = BatchNormalization()(x)
    # x = Normalization2D(str_axis='batch')(x)
    # print(np.array(x).shape)
    # x = Reshape((2, 64))(x)
    # x = GRU(64)(x)
    x = Flatten()(x)
    x = Dense(32, activation="relu", kernel_regularizer=l2(l2_lambda))(x)
    x = BatchNormalization()(x)

    return x


def single_channel_resnet(my_input, num_filters=64, num_res_blocks=4, cnn_per_res=3,
                          kernel_sizes=[8, 5, 3], max_filters=128, pool_size=3, pool_stride_size=2):

    #my_input = Input(shape=input_shape)
    # my_input = input_shape
    # my_input = ks.expand_dims(my_input, axis=2)

    for i in np.arange(num_res_blocks):
        if (i == 0):
            block_input = my_input
            x = BatchNormalization()(block_input)
        else:
            block_input = x

        for j in np.arange(cnn_per_res):
            x = Conv1D(num_filters, kernel_sizes[j], padding='same')(x)
            x = BatchNormalization()(x)
            if (j < cnn_per_res - 1):
                x = Activation('relu')(x)

        is_expand_channels = not (my_input.shape[0] == num_filters)

        if is_expand_channels:
            res_conn = Conv1D(num_filters, 1, padding='same')(block_input)
            res_conn = BatchNormalization()(res_conn)
        else:
            res_conn = BatchNormalization()(block_input)

        x = add([res_conn, x])
        x = Activation('relu')(x)

        if (i < 5):
            x = AveragePooling1D(pool_size=pool_size, strides=pool_stride_size)(x)

        num_filters = 2 * num_filters
        if max_filters < num_filters:
            num_filters = max_filters

    return my_input, x


def raw_signals_deep_ResNet(input, UseDerivative=False):
    fs=125

    inputs = []
    l2_lambda = .001
    channel_outputs = []
    num_filters = 32

    X_input = Input(shape=input)

    if UseDerivative:
        # fs = tf.constant(fs, dtype=float)
        X_dt1 = Lambda(diff, arguments={'fs': fs})(X_input)
        X_dt2 = Lambda(diff, arguments={'fs': fs})(X_dt1)
        X = [X_input, X_dt1, X_dt2]
    else:
        X = [X_input]

    num_channels = len(X)

    for i in np.arange(num_channels):
        channel_resnet_input, channel_resnet_out = single_channel_resnet(X[i], num_filters=num_filters,
        num_res_blocks=4, cnn_per_res=3, kernel_sizes=[8, 5, 5, 3], max_filters=64, pool_size=2, pool_stride_size=1)
        channel_outputs.append(channel_resnet_out)
        inputs.append(channel_resnet_input)

    spectral_outputs = []
    num_filters = 32
    for x in inputs:
        spectro_x = mid_spectrogram_LSTM_layer(x)
        spectral_outputs.append(spectro_x)

    # concatenate the channel specific residual layers

    # print("Num Channels: ", num_channels)

    if num_channels > 1:
        x = concatenate(channel_outputs, axis=-1)
    else:
        x = channel_outputs[0]

    x = BatchNormalization()(x)
    x = GRU(65)(x)
    # x = Flatten()(x)
    x = BatchNormalization()(x)

    # join time-domain and frequnecy domain fully-conencted layers
    if num_channels > 1:
        s = concatenate(spectral_outputs, axis=-1)
    else:
        s = spectral_outputs[0]

    # s = Flatten()(s)
    #     x = Dense(128,activation="relu",kernel_regularizer=l2(l2_lambda)) (x)
    s = BatchNormalization()(s)
    # LETS DO OVERFIT
    x = concatenate([s, x])
    x = Dense(32, activation="relu", kernel_regularizer=l2(l2_lambda))(x)
    x = Dropout(0.25)(x)
    x = Dense(32, activation="relu", kernel_regularizer=l2(l2_lambda))(x)
    x = Dropout(0.25)(x)
    #output = Dense(2, activation="relu")(x)
    x = Flatten()(x)
    X_SBP = Dense(1, activation='linear', name='SBP')(x)
    X_DBP = Dense(1, activation='linear', name='DBP')(x)

    model = Model(inputs=X_input, outputs=[X_SBP, X_DBP], name="Slapnicar_Model")
    # model = multi_gpu_model(model, gpus=2)
    # optimizer = optimizers.Adadelta()
    # loss = ks.keras.losses.mean_absolute_error
    # model.compile(optimizer=optimizer, loss=loss, metrics=['mae', 'mae'])
    print(model.summary())
    # plot_model(model=model, to_file='lstm_model.png', show_shapes=True, show_layer_names=True)
    return model


def one_chennel_resnet(input_shape,num_filters=16,num_res_blocks = 5,cnn_per_res = 3,
                        kernel_sizes = [3,3,3], max_filters = 64, pool_size = 3,
                        pool_stride_size = 2,num_classes=8):
    my_input  = Input(shape=(input_shape))
    for i in np.arange(num_res_blocks):
        if(i==0):
            block_input = my_input
            x = BatchNormalization()(block_input)
        else:
            block_input = x
        for j in np.arange(cnn_per_res):
            x = Conv1D(num_filters, kernel_sizes[j], padding='same')(x)
            x = BatchNormalization()(x)
            if(j<cnn_per_res-1):
                x = Activation('relu')(x)
        is_expand_channels = not (input_shape[0] == num_filters)
        if is_expand_channels:
            res_conn = Conv1D(num_filters, 1,padding='same')(block_input)
            res_conn = BatchNormalization()(res_conn)
        else:
            res_conn = BatchNormalization()(block_input)
        x = add([res_conn, x])
        x = Activation('relu')(x)
        if(i<5):
            x = MaxPooling1D(pool_size=pool_size,strides =pool_stride_size)(x)
        num_filters = 2*num_filters
        if max_filters<num_filters:
            num_filters = max_filters
    return my_input,x


def one_chennel_resnet_2D(input_shape, input_layer, num_filters=16,num_res_blocks = 5,cnn_per_res = 3,
                        kernel_sizes = [8, 5, 3], max_filters = 64, pool_size = (3,3),
                        pool_stride_size = 2, num_classes=8):
    kernel_sizes = [(8, 1), (5, 1), (3, 1)]
    my_input = input_layer
    for i in np.arange(num_res_blocks):
        if(i==0):
            block_input = my_input
            x = BatchNormalization()(block_input)
        else:
            block_input = x
        for j in np.arange(cnn_per_res):
            x = Conv2D(num_filters, kernel_sizes[j], padding='same')(x)
            x = BatchNormalization()(x)
            if(j<cnn_per_res-1):
                x = Activation('relu')(x)
        is_expand_channels = not (input_shape[0] == num_filters)
        if is_expand_channels:
            res_conn = Conv2D(num_filters, (1,1), padding='same')(block_input)
            res_conn = BatchNormalization()(res_conn)
        else:
            res_conn = BatchNormalization()(block_input)
        x = add([res_conn, x])
        x = Activation('relu')(x)
        if(i<5):
            x = MaxPooling2D(pool_size=pool_size,strides =pool_stride_size)(x)
        num_filters = 2*num_filters
        if max_filters<num_filters:
            num_filters = max_filters
    return my_input,x


def spectro_layer_mid(input_x,sampling_rate, ndft=0, num_classes=8):
    l2_lambda = .001
    if(ndft == 0):
        n_dft= 128
    else:
        n_dft = ndft
    # n_dft = 64
    n_hop = 64
    fmin=0.0
    fmax=sampling_rate//2

    x = Permute((2,1))(input_x)
    x = STFT(n_fft=n_dft, hop_length=n_hop, output_data_format='channels_last')(x)
    x = Magnitude()(x)
    #x = MagnitudeToDecibel()(x)
    #x = BatchNormalization()(x)    # x = Normalization2D(str_axis='batch')(x)
    channel_resnet_input,channel_resnet_out= one_chennel_resnet_2D((625, 1), x, num_filters=64,
                    num_res_blocks = 6,cnn_per_res = 3,kernel_sizes = [3,3,3,3],
                    max_filters = 32, pool_size = 1,
                    pool_stride_size =1,num_classes=8)
    channel_resnet_out = BatchNormalization()(channel_resnet_out)

    # x = Reshape((10, 65))(x)
    # x = GRU(65)(x)

    return channel_resnet_out
\end{python}

\subsubsection*{LSTM}

\begin{python}
    from tensorflow.keras.layers import Input, LSTM, Dense, Bidirectional, Conv1D, ReLU

from tensorflow.keras import Model

def define_LSTM(data_in_shape, UseDerivative=False):
    X_input = Input(shape=data_in_shape)

    fs = 125

    if UseDerivative:
        dt1 = (X_input[:,1:] - X_input[:,:-1])*fs
        dt2 = (dt1[:,1:] - dt1[:,:-1])*fs

        dt1 = tf.pad(dt1, tf.constant([[0,0],[0,1],[0,0]]))
        dt2 = tf.pad(dt2, tf.constant([[0,0],[0,2],[0,0]]))

        X = tf.concat([X_input, dt1, dt2], axis=2)
    else:
        X=X_input


    X = Conv1D(filters=64, kernel_size=5, strides=1, padding='causal', activation='relu')(X)
    X = Bidirectional(LSTM(128, return_sequences=True))(X)
    X = Bidirectional(LSTM(128, return_sequences=True))(X)
    X = Bidirectional(LSTM(64, return_sequences=False))(X)
    X = Dense(512, activation='relu')(X)
    X = Dense(256, activation='relu')(X)
    X = Dense(128, activation='relu')(X)

    X_SBP = Dense(1, name='SBP')(X)
    X_DBP = Dense(1, name='DBP')(X)

    model = Model(inputs=X_input, outputs=[X_SBP, X_DBP], name='LSTM')

    return model
\end{python}

\subsubsection*{Transformer encoder}
\begin{python}
from tensorflow import keras
from tensorflow.keras import layers

def transformer_encoder(inputs, head_size, num_heads, ff_dim, dropout=0):
    # Normalization and Attention
    x = layers.LayerNormalization(epsilon=1e-6)(inputs)
    x = layers.MultiHeadAttention(
        key_dim=head_size, num_heads=num_heads, dropout=dropout
    )(x, x)
    x = layers.Dropout(dropout)(x)
    res = x + inputs

    # Feed Forward Part
    x = layers.LayerNormalization(epsilon=1e-6)(res)
    x = layers.Conv1D(filters=ff_dim, kernel_size=1, activation="relu")(x)
    x = layers.Dropout(dropout)(x)
    x = layers.Conv1D(filters=inputs.shape[-1], kernel_size=1)(x)
    return x + res


def define_encoder(
    input_shape,
    head_size,
    num_heads,
    ff_dim,
    num_transformer_blocks,
    mlp_units,
    dropout=0,
    mlp_dropout=0,
    UseDerivative=False
):
    inputs = keras.Input(shape=input_shape)
    X_input = inputs
    fs = 125

    if UseDerivative:
        dt1 = (X_input[:,1:] - X_input[:,:-1])*fs
        dt2 = (dt1[:,1:] - dt1[:,:-1])*fs

        dt1 = tf.pad(dt1, tf.constant([[0,0],[0,1],[0,0]]))
        dt2 = tf.pad(dt2, tf.constant([[0,0],[0,2],[0,0]]))

        x = tf.concat([X_input, dt1, dt2], axis=2)
    else:
        x=X_input

    for _ in range(num_transformer_blocks):
        x = transformer_encoder(x, head_size, num_heads, ff_dim, dropout)

    x = layers.GlobalAveragePooling1D(data_format="channels_first")(x)
    for dim in mlp_units:
        x = layers.Dense(dim, activation="relu")(x)
        x = layers.Dropout(mlp_dropout)(x)
    
    # outputs = layers.Dense(1)(x)

    X_SBP = layers.Dense(1, name='SBP')(x)
    X_DBP = layers.Dense(1, name='DBP')(x)

    model = Model(inputs=inputs, outputs=[X_SBP, X_DBP], name='Encoder')
    # return keras.Model(inputs, outputs)
    return model
\end{python}
