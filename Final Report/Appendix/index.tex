\section{Appendix}
\begin{comment}
   The appendices contain information
that is peripheral to the main body of the report.
Information typically included are things like
program listings, complex circuit diagrams,
tables, proofs, graphs or any other material
which would break up the theme of the text if it
appeared in situ. Large program listings may be
submitted with the report although it is
preferable either to provide them on CD, or to
cite details of a suitable accessible cloud
repository containing the material. Where CDs
are used you must prepare two CDs, one for each
paper copy of the report.\\ \newline \noindent Please use 
appendices as necessary for material that
is tangentially relevant, or necessary to preserve
project value, but that you do not expect Examiners to
read. Note that software projects with significant code
should normally provide electronic versions of the
code on USB stick, CDROM, or cloud repository. In
that case the Appendix should state where the code
may be found. \textbf{The Appendix should normally
include, or refer to, all the technical details needed
for another user to continue code development}.
Typical items that should not be in the main report,
but should (possibly - since Appendices are not
normally read this is a matter of judgement) be in
appendices:
\begin{itemize}
    \item Source code. Note that code (if long) should
    also be available in some electronic form, e.g. a github repository, CD, or USB 
    stick. Software
    projects must provide access to the source
    code in some form. The Appendix will then
    reference this.
    \item Test data sets (again, for large volumes of
    data an electronic form would be more
    appropriate). The report itself will contain
    concise summaries of the test data in a
    human readable form.
    \item Raw results. Tables of results in unreadable
    form should not be put in the main report, but
    if needed may be put in an Appendix.
    \item Related material possibly but not directly
    relevant to the project work. E.g. manuals of
    test equipment used. (There is no
    requirement to include such, but in some
    cases, where they have some tangential
    relevance, it might be appropriate)
\end{itemize}
 
\end{comment}
\subsection{FYP Mission Statement (correct to June 2022)}
'\emph{Ambulatory blood pressure monitoring has become increasingly 
relevant due to the advancement of wearable technology. 
Modalities such as Electrocardiogram (ECG) and Photoplethysmography 
(PPG) have provided an indirect method of blood pressure estimations 
compared to a traditional blood-pressure monitor. Although algorithms 
exist for calculating blood pressure with ECG and PPG, it is vital 
that their computational complexity is minimal, whilst maintaining 
accuracy, due to the power limitations of wearable technology. The 
goal of this project is to establish the tradeoffs between using 
PPG and ECG to quantify blood pressure, in the context of wearable
 technology, where power must be kept to a minimum. This project is
  ideal for students interested in signal processing, who have 
  excellent programming skills in Matlab.}'


\subsection{Health standards requirements for blood pressure estimation}
\begin{itemize}
    \item The Advancement of Medical Instrumentation (AAMI) standard requires a mean BP difference of $\le 5$ mmHg with a standard deviation of $\le 8$ mmHg against auscultatory reference measurement
    \item Significant variation in BP measurements($> 12$ mmHg systolic or $> 8$ mmHg diastolic) from the validated reference device is an exclusion criterion in the AAMI protocol \cite{Bard2019}
    \item The European Society of Hypertension (ESH) protocol requires that the majority of subjects have investigational BP readings within $\le 5$ mmHg of the reference measurement.
\end{itemize}

\subsection{Complete FYP Gantt chart}