\section{Evaluation of results}

\begin{comment}
    This Chapter (or possibly section of the conclusions) is
distinct from your results. It must contain your critical
evaluation of your work as compared to previous
analysis, algorithms, products, and when related to
your original objectives. To what extent have your
original objectives been fulfilled? If they have
changed, what is your rationale for this? What are the
advantages, disadvantages of your approach
compared with related work? How does the scope of
your work differ from related work? Examiners expect
your project report to show evidence of your ability to
think as an engineer, and that includes the ability to
critically reflect on your own work and evaluate its
significance.\\ \newline \noindent Material here will compare project outcomes with
initial objectives and requirements captured. Usually
your Interim Report will contain these. Where these
have changed significantly over the course of the
project this should be explained and reasons given.
This section should not require examiners to read
your Interim Report, and will not reference it. Changes
between final and initial objectives should be
explained in a self-contained manner.\\ \newline \noindent Note that here you will reference and summarise,
rather than repeat, your description of Requirements
Capture earlier in the Final Report. 
\end{comment}


\begin{itemize}
    \item All models except AlexNet have acceptable validation errors
    \item RNN/LSTM perform best on this data, not Transformer as expected (for no derivatives)
    \item Derivative?
\end{itemize}