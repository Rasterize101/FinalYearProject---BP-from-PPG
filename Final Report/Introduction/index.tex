\section{Introduction}
\begin{comment}
   This is one of the most important
components of the report. It should begin with a
clear statement of what the project is about so
that the nature and scope of the project can be
understood by a lay reader. Further information can be found in the separate
Final Report Structure and Contents 
document.\\ \newline \noindent The introduction should set the scene and give a high-level problem statement/specification, so that after
reading the introduction the reader understands
roughly what the problem is and what you intend to
do about it. Is the idea to write software, or develop
an algorithm, or produce hardware, or something
else?
You should then highlight and summarise the most
interesting or important questions or problems that
your project addresses, and the broader context in
which those questions or problems are situated.
Finally, you must briefly introduce the structure of
report (what you will cover in which chapters and how
these relate to each other). You don't need to go into
any detail, the aim is to make sure the reader has an
idea about what will be discussed and in what order. 
\end{comment}


\subsection{Motivation}
Cardiovascular disease is one of the main causes of death around the world. 
High blood pressure (BP), which is also known as hypertension, is a common 
condition which can be a cause of cardiovascular disease \cite{Sharma2017}. 
According to the World Health Organization (WHO), the mortality rate due 
to hypertension is 9.4 million per year and it causes 55.3\%  of  total  
deaths  in  cardiovascular patients \cite{Janjua2017}. If hypertension is 
detected early and prevented, this will greatly lower the number of deaths 
associated with cardiovascular 
diseases \cite{Janjua2017}. \\ \newline \noindent Recent developments in 
technology have made wearable sensors, such as Electrocardiogram (ECG) and 
Photoplethysmography (PPG) sensors significantly more popular in today's 
world. These sensors provide real-time 24 hour monitoring of the human 
bodily function. Hence there is great potential in using these sensors to 
diagnose medical conditions, such as hypertension, in real-time, thus 
helping to save lives \cite{Simjanoska20182}. Ambulatory BP monitoring 
is seen as a promising method for detecting early symptoms of 
hypertension \cite{Kario2021}. There is a lot of existing research to 
predict ambulatory BP using methods which are cuff-less, continuous 
and non-invasive \cite{Zaki2018}. Hence wearables are seen as a 
viable option for this. ECG and PPG sensors have been discovered 
to be a potential estimator of blood pressure that cause minimal 
harm to patients compared to existing cuff-based 
methods \cite{Malikeh2019} \cite{Bard2019}.

\subsection{High-level problem statement}
\textcolor{red}{Needs rewriting. Aim: Your intention/what you hope to achieve.}

\textcolor{red}{Objectives: Statements of measurable outcomes/What will you be doing to achieve the aim/desire outcome. (this is the work you're going to do)}
%-----
\textcolor{red}{The aim is not to implement and evaluate different techniques. The aim is to estimate cuffless BP using PPG for wearable technology purpose.}
%-----
\textcolor{red}{I advise you to have a clear aim and clear objectives. Objectives can be listed as bullets or numerated. These objectives will reappear in your conclusion where you state if you completed them or not.}

Based on the provided motivation, the main aim of this Final Year Project (FYP) is to estimate 
cuff-less blood pressure values using ECG and/or PPG signals, so that this estimation process 
can be integrated onto future wearable technology devices. Hence, these are the following objectives 
for this Final Year Project (FYP):

\begin{itemize}
    \item Conduct a literature review to assess what are the most popular methods for estimating cuff-less BP values and to decide on the most feasible implementation for this FYP\\
    \item Develop a novel algorithm in the Python programming language to estimate cuffless blood pressure\\
    \item Assess the performance of this algorithm against existing methods\\
    \item Conclude whether this method is feasible for future wearable technology products
\end{itemize}

\subsection{Overview of work}
This section provides a chronological overview of the work that will be done in this FYP. The comprehensive overview 
of the work is provided in the Gantt Chart in the Appendix.

\subsubsection{Autumn Term 2021}
The main tasks for this term were classified under the theme of \textbf{Research and Understanding}. The main tasks were as follows:
\begin{itemize}
    \item Searching for all background knowledge required to understand the motivation and objectives of this project
    \item Investigation into existing experimentations conducted with wearable technologies for estimating blood pressure
    \item Research into signal denoising techniques and machine learning based methods for estimating BP
    \item Familiarisation with the PhysioNet MIMIC database
    \item Conducting a literature survey in order to assess what is the best method for the cuffless estimation of blood pressure
\end{itemize}

\subsubsection{Spring Term 2022}
The main tasks for this term were classified under the theme of \textbf{Implementation of chosen methods}. The main tasks were as follows:
\begin{itemize}
    \item The implementation of Convolutional Neural Network (CNN) architectures    
    \item The implementation of a Recurrent Neural Network - Long Short Term Memory (RNN - LSTM) architecture
    \item The implementation of a Transformer Encoder architecture
    \item Testing of the architectures on the MIMIC database using different parameters (window length, ...?)
\end{itemize}

\subsubsection{Summer Term 2022}
The main tasks for this term were classified under the theme of \textbf{Performance Testing}. The main tasks were as follows:
\begin{itemize}
    \item Finetuning of the Transformer Encoder architecture
    \item Graphical comparisons of the different architectures using error metrics
    \item Interpretation of results and conclusions 
\end{itemize}




\begin{comment}
    \subsection{Report structure}
    This section briefly introduces the structure of the main body of the report. The aim is to explain what will be covered in each chapter and how
these chapters relate to each other. The list is detailed below:
\begin{itemize}
    \item \textbf{Background Information}. This chapter will first illustrate the 
    literature review conducted and highlight what is the most feasible method for BP 
    estimation. As a result of this review, the necessary background information will be 
    discussed. The main background topics discussed will be medical background and machine 
    learning principles.
    \item \textbf{Analysis and Design}. This chapter will discuss the methodology used to
    \item  \textbf{Implementation}. This chapter will describe what work has been done in regards to the chosen method for estimating blood pressure. In addition, a mathematical overview of the chosen implementation will be discussed in detail. Finally, the justification for using Python as the sole programming language for this project will be given.
    \item \textbf{Testing Plan}. This chapter will describe the testing plan for the project.
    \item \textbf{Overview of Results}. This chapter will discuss the results of the implementation.
    \item \textbf{Evaluation of Results}. This chapter will discuss the evaluation of the results.
    \item \textbf{Evaluation plan}. This chapter will detail how the project deliverables will be evaluated. In addition, the chapter will show the estimated timeline and possible extensions.
    \item \textbf{Conclusions and Further Work}. This chapter will discuss the conclusions of the project and provide a list of possible future work.
\end{itemize}
\end{comment}
