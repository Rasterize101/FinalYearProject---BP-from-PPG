\section{Introduction}

\subsection{Motivation}
Cardiovascular disease is one of the main causes of death around the world. 
High blood pressure, which is also known as hypertension, is a common 
condition which can be a cause of cardiovascular disease \cite{Sharma2017}. 
According to the World Health Organization (WHO), the mortality rate due 
to hypertension is 9.4 million per year and it causes 55.3\%  of  total  
deaths  in  cardiovascular patients \cite{Janjua2017}. If hypertension is 
detected early and prevented, this will greatly lower the number of deaths 
associated with cardiovascular 
diseases \cite{Janjua2017}. \\ \newline \noindent Ambulatory blood pressure monitoring 
is seen as a promising method for detecting early symptoms of 
hypertension \cite{Kario2021}. There is a lot of existing research to 
predict ambulatory blood pressure using methods which are cuffless, continuous 
and non-invasive \cite{Zaki2018}. Recent developments in 
technology have made wearable sensors, such as Electrocardiogram (ECG) and 
Photoplethysmography (PPG) sensors significantly more feasible to be used 
for research into the long-term monitoring of blood pressure. These sensors cause minimal 
harm to patients compared to existing cuff-based 
methods \cite{Malikeh2019} \cite{Bard2019}. These sensors can also provide real-time 24 hour monitoring of the human 
body and have shown to be correlated with the behaviour of how the heart 
pumps blood around the body \cite{Bard2019}. Hence there is great potential in using these sensors in wearable devices to 
diagnose medical conditions, such as hypertension, in real-time, thus 
helping to save lives \cite{Simjanoska20182}. 

\subsection{High-level problem statement}
Based on the provided motivation, the main aim of this Final Year Project (FYP) is to estimate 
cuff-less blood pressure values using ECG and/or PPG signals, so that this estimation process 
can be integrated onto future wearable technology devices. Hence, these are the objectives that will be assessed at the end of this report:

\begin{itemize}
    \item Conduct a literature review to assess what are the best performing methods for estimating cuffless blood pressure values and to decide on the most feasible implementation for this FYP
    \item Develop a novel algorithm in the Python programming language to estimate cuffless blood pressure
    \item Assess the performance of this algorithm against existing methods
    \item Conclude whether this method is feasible for future wearable technology products
\end{itemize}

\subsection{Overview of work}
This section provides a chronological overview of the work that will be done in this FYP. The comprehensive overview 
of the work is provided in the Gantt Chart \cite{Gantt}.

\subsubsection{Autumn Term 2021}
The main tasks for this term were classified under the theme of \textbf{Research and Understanding}. The main tasks were as follows:
\begin{itemize}
    \item Searching for all background knowledge required to understand the motivation and objectives of this project
    \item Investigation into existing experimentations conducted with wearable technologies for estimating blood pressure
    \item Research into signal denoising techniques and machine learning based methods for estimating BP
    \item Familiarisation with the PhysioNet MIMIC database
    \item Conducting a literature survey in order to assess what is the best method for the cuffless estimation of blood pressure
\end{itemize}

\subsubsection{Spring Term 2022}
The main tasks for this term were classified under the theme of \textbf{Implementation of chosen methods}. The main tasks were as follows:
\begin{itemize}
    \item The implementation of Convolutional Neural Network (CNN) architectures    
    \item The implementation of a Recurrent Neural Network - Long Short Term Memory (RNN - LSTM) architecture
    \item The implementation of a Transformer Encoder architecture
    \item Testing of the architectures on the MIMIC database using different parameters (window length, PPG derivative features)
\end{itemize}

\subsubsection{Summer Term 2022}
The main tasks for this term were classified under the theme of \textbf{Performance Testing}. The main tasks were as follows:
\begin{itemize}
    \item Finetuning of the Transformer Encoder architecture
    \item Graphical comparisons of the different architectures using error metrics
    \item Interpretation of results and conclusions 
\end{itemize}