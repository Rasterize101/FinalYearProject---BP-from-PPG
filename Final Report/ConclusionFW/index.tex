\section{Conclusions and Further Work}
All good projects conclude with an objective evaluation of 
the project's successes and failures and
suggestions for future work which can take the
project further. It is important to understand that
there is no such thing as a perfect project. Even
the very best pieces of work have their
limitations and you are expected to provide a
proper critical appraisal of what you have done.
Your assessors are bound to spot the limitations
of your work and you are expected to be able to
do the same. Further information can be found in the separate
Final Report Structure and Contents document.\\ \newline \noindent How successful have you been? What have you
achieved? How could the work be taken further given
more time (perhaps by another student next year)? It
is important here to identify positively what is
worthwhile in your work. At the same time, honesty,
and a clear description of the limits of your work, is
equally important. It is often most appropriate to
describe work you did not have time to complete as
further work.\\ \newline \noindent Your readers will not be clear where, in your long
report, are your most significant achievements. In the
conclusions you must summarise this, referring as
necessary to other sections for more detail.
\begin{itemize}
    \item What design choices did you have along the
    way, and why did you make the choices you
    made?
    \item What was the most difficult and/or clever part
    of the project?
    \item Why was it difficult?
    \item How did you overcome the difficulties?
    \item Did you discover or invent anything novel?
    \item What did you learn?
\end{itemize}\noindent Note that “difficult” does not necessarily mean the
thing that took you the longest amount of time. Note
also that the conclusions must concisely summarise
this material, and refer to other sections for the
details.
