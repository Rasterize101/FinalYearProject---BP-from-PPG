\section{Testing Plan}

\begin{comment}
    Describe your Test Plan - how the program or system
was verified. Put the actual test results in an Appendix if they are repetitive but relevant. Detailed test data
may be omitted from the report if not relevant,
however an accurate summary of tests should be
included in the Report itself. Sometimes non-working
designs are described in project reports as though
they work, when in reality they don't, or only partially
work. Therefore a precise description of what works
and how this has been established is important.
Examiners may try to compile, use, or test
deliverables themselves (even after your report is
submitted), and your report should accurately reflect
the state of the project.
    
\end{comment}

In this chapter, the aims are to:

\begin{itemize}
    \item Highlight what neural network architectures are being tested for this FYP
    \item State what parameters were used for all experiments
    \item Explain how the results are being calculated
    \item Highlight any other measures that are calculated
\end{itemize}\noindent The aim of this project 
is to assess the best performing neural network 
for the cuffless estimation of blood pressure values 
from PPG signals. As a result of the findings of the Literature review, 
it has been made clear that there are three distinct groups of neural networks 
that have been able to produce accurate estimates, according to the AAMI/ESH standards. These three groups are:

\begin{itemize}
    \item \textbf{Convolutional Neural Networks (CNNs)}: AlexNet, ResNet and ResNet- Leave One Subject Out (LOSO)
    \item \textbf{Recurrent Neural Networks (RNNs)}: Bi-directional Long Short Term Memory (Bi-LSTM) 
    \item \textbf{Transformer ANNs}: Transformer Encoders
\end{itemize}\noindent The following parameters are used during the training, validation and testing processes of the models on the MIMIC I database subset:

\begin{itemize}
    \item The training:test:validation dataset split is 50:25:25
    \item Loss function: Mean Squared Error
    \item Optimiser: Adam
    \item Learning rate: 0.005 arbitary units (AU)
    \item Batch size: 128 samples
    \item Window length: 1 second
    \item Number of epochs: 100
    \item Metrics: Mean Absolute Error (MAE) of the Systolic and Diastolic blood pressure values, measured in millimetres of Mercury (mmHg)
    \item For all testing performed in Google Colaboratory, the GPU being used to compute
    inference times was the Tesla T4 Persistence-M GPU
    \item In order to produce reproducible results, each of the five models were run 3 times, and the average of the minimum MAE for SBP and DBP is presented in the tables shown in the next chapter
\end{itemize}\noindent To conclude this chapter, the overview of the testing strategy has now been clearly defined. Hence, it is now suitable to proceed with the collection of results.