\section{Analysis and Design}
If your project involves designing a system, give a
good high-level overview of your design.\\ \newline \noindent In many projects, the final design is different from
that originally envisaged. If the differences are
interesting, write about them, and why the changes
were made. Discoveries during the project may have changed the direction of work, or invalidated prior
work, in which case you get credit for the design
process, if it is principled, as well as the end product.\\ \newline \noindent If your design was not implemented fully, describe
which parts you did implement, and which you didn't.
If the reason you didn't implement everything is
interesting (eg it turned out to be difficult for
unexpected reasons), write about it.\\ \newline \noindent Note that the Project Report is written at the end of
project work and must describe the project work, but
need not do this chronologically. Often the best
description of design, in retrospect, is far from the
way in which you developed it. Where the evolution
of ideas is interesting or relevant it can be described,
as above, but otherwise the order in which things
were done need not be documented.\\ \newline \noindent The Examiners are just as interested in the
engineering process you went through in performing
your project work as the results you finally produced.
So make sure your report identifies when design
choices have to be made, what were the possibilities,
and why you made the particular choices and
decisions that you did. They are looking for principled
rational arguments and for critical assessment.
Engineering involves trade-offs and the reasons for a
design decision may be various, and may in some
cases be out of your control. Explicit understanding of
this, and the ability to communicate it, is important.
