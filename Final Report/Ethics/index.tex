\section{Ethical, Legal and Safety Plan}

\subsection{Ethical considerations}
This section discusses the ethical implications of this project through the perspective 
of the Four Pillars of Medical Ethics: beneficence, non-maleficence, autonomy, and 
justice \cite{MedicalEthics}. 

\subsubsection{Beneficence}
Beneficence is the duty to do good. During the process of this project, it must be 
ensured that actions must be taken with the intent of helping patients. The purpose 
of future wearable technologies is to help people to detect early risk factors of 
heart-related pathologies. \\ \newline \noindent The issues that may arise would 
be due to the occurrence of false positives and negatives in the data, such as 
a normotensive result for a person who is actually hypertensive. Ideally, a 
blood pressure monitor which is cuffless should operate accurately over a 
range of blood pressure values after having been calibrated. However this 
has not been tested according to the established medical standards, such 
as AAMI and ESH. Measurements that are inaccurate will naturally cause a 
false sense of reassurance with the patients, unnecessary health care 
utilization, or improper administration of hypertension medications. 
An incorrect diagnosis also causes unnecessary psychological stress and 
the unnecessary reactions to the negative side effects of prescribed 
medication. For example, if there is a range of $\pm 5$ mmHg in error, 
this could lead to approximately twenty-seven million hypertensive and 
twenty-one million normotensive patients being classified 
incorrectly \cite{Bard2019}. This can only be minimised through robust 
tuning of the parameters in the neural networks used.

\subsubsection{Non-maleficence}
Non-maleficence is the duty to not harm. Through the recording of ECG and PPG 
data, it must be ensured that any actions must not be harmful to the patients. 
An instance where the product could harm patients would be if the models 
falsely classify a patient to be normotensive, whereas the patient indeed 
is hypertensive in real life. If test accuracy could be proven, it would 
be possible to see if a model is provably 100\% accurate, or otherwise 
justify to an Ethics Board that the risks would be worth the reward. However, 
with neural networks of this complexity, it is practically impossible to
 prove any estimate of test accuracy. 

\subsubsection{Autonomy}
Autonomy is the respect of patients' freedom of choice and consent. During 
the data acquisition process, it must be ensured that there are no actions 
taken that are non-consensual to the patients. Those concerned should also 
be informed about the benefits and risks of using a particular wearable 
technology product for BP measurement, such as the validation accuracy 
and inference time, allowing them to make an informed decision. \\ \newline \noindent Another 
general issue of using deep neural networks is that it is very difficult to explain how 
models obtain their results. It is impossible to explain what features an individual 
or end-to-end model detects and what any of their weights correspond to. 
This can be an issue if the user wants an explanation as to why they have 
been classified wrongly as hypertensive or hypotensive, as they should have the right to know.

\subsubsection{Justice}
Justice is the principle of fairness towards all patients and the idea of treating 
all people equally and equitably. The product must ensure it does not discriminate 
between patients and medical staff on any basis. An obvious concern is 
whether the dataset used to train the models are balanced in representing 
both gender and ethnicity. Although the dataset was created taking such biases 
into account, it is difficult to overcome subconscious biases and availability 
of a balanced dataset. This is important as it ensures that future products in 
wearable technologies have both a sustainable customer base but is also sustainable 
in today's social ecosystem, as it is actively standing against racism.

\subsection{Legal considerations}
Legally this project poses no risk. The Python programming language and its required 
libraries are Open Source, including the version which I intend to use (3.10). Python 
has a license agreement under the Python Software Foundation (PSF). Hence it is a tool 
that can be legally used freely for the purposes of this project. \cite{Python}. In 
addition, the ECG and PPG data is provided by a website called Physionet. PhysioNet 
is a data repository of medical signals which are free to use. The repository is 
managed by the MIT Laboratory for Computational 
Physiology \cite{Goldberger2000} \cite{Physionet}. As this code will be published 
online, the produced code could be plagiarised in the future. However, I myself will 
bear no responsibility in the distribution of my code to others.

\subsection{Safety considerations}
With regards to this project, there are no safety issues to consider, as all experimentation 
will be conducted online at home. It is critical that necessary measures are taken such 
that strains or injuries from desk work are prevented. This can easily be done by following 
NHS guidelines. \\ \newline \noindent As highlighted in Chapter 3, there are a 
few points to consider if this work becomes implemented in a future wearable 
technology product. Firstly, it is important to remember these are only simulations. 
Training a machine learning model on Python will not accurately reflect the behavior 
in real life. Secondly, any results discussed in this report will not be officially 
recognised by the AAMI or ESH protocols. This is something that future active projects 
will need to consider, in order to have substantial results.